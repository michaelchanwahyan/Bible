\chapter*{編者結語}

余編彙聖卷合本耗時一年.
此念緣余決志後所得之異象.
\newline
\newline
昔時.余返大埔永光堂半春秋.
藉神恩.堂聚會共鳴心坎.回響自幼兩載經歷.
藉神恩.得體察神作為.親歷其事致使信心愈大.而得決志一刻.
有恩於靈感召.靈賜勇為.故善擇小組長洲之旅.取長洲綠柳亭之涼蔭.請眾弟兄妹與共.行決志禱告.
決志其樂為大.其志歷歷在目.
\newline
\newline
今時.華文聖卷之弊於文譯.乃譯之三難:順.達.雅也.
況希伯來.希利尼二文之意譯屢為華文所限.聖卷之義難盡為華文忠傳.
余見及此.遂使編程之技.為聖卷編集眾中.英普及譯本.旁以當代普及漢字重塑古漢字.
余願詳參各本之譯.以愈明瞭聖卷之道.同授當今華人基督徒及眾懷志之士以共參詳.
有恩於靈保守.並下恩賜.余繼往是編彙工程而無逢大礙.
\newline
\newline
是時.略距長洲綠柳亭決志日一年.淡顯微義.
\newline
\newline
詩百又廿七載:
\newline
\newline
\begin{center}
    非耶和華建屋宇. 則匠人之經營徒. 
\newline
    非耶和華衛城邑. 則守者之儆醒徒.
\newline
\newline
\end{center}
余獻是聖卷合本予華人社區.願為福音流通之器.願獻斯微材為祭榮耀上帝.
\newline
\newline
阿門.
\newline
\newline
二零一八年七月一日~~ 主日
\newline
\newline
米迦勒 ~~ 書於香港
